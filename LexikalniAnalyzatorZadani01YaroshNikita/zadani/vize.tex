\section {Vize}
\label{sec:vize}

Program \uv{\textbf{Lexikální analyzátor}} bude umožňovat zpracování textu ze souboru a zobrazení výsledků v podobě seznamu tokenů na monitoru.\\

Proč takové zadání?\\

Pro zpracování textu a převod do dat v počítači byla v minulém semestru využívána třída Scanner. 
Její instance umožňovala bohaté zpracování textu ze vstupního zařízení. 
Její nevýhodou bylo, že se muselo předpokládat jaký druh informace v textu bude. 
Když se očekávalo celé číslo musela se použít metoda \lstinline|nextInt()| a když reálné číslo tak \lstinline|nextDouble()| nebo  \lstinline|nextfloat()|.
Před tím se mohlo zeptat, zda takový formát dat je v textu připraven ke zpracování. 
K tomu sloužily metody \lstinline|hasNextInt()|, \lstinline|hasNextDouble()|, \lstinline|hasNextFloat()| a tak podobně. 
To je někdy nepraktické a v případě jednoduché rozboru vstupního textu je výhodnější si naprogramovat vlastní řešení. 
K takovému řešení se obvykle říká lexikální analyzátor (anglicky scanner).\\

Lexikální analýza, kterou provádí lexikální analyzátor, rozděluje posloupnost znaků na lexikální elementy (například identifikátory, čísla, klíčová slova, operátory, a pod.). 
Tímto způsobem překladače zpracovávají naše zdrojové kódy do tzv. tokenů a ty se dál poskytují k dalšímu zpracování v syntaktickém analyzátoru.
S tímto se někteří studenti setkají na navazujícím magisterském studiu v předmětu kompilátory nebo v teorii jazyků.\\

My samozřejmě nebudeme realizovat kompilátor, ale pokusíme se o naprogramování jednoduché lexikální analýzy v podobě malého lexikálního analyzátoru.\\

Náš program rozebere vstupní text na výstupní tokeny v podobě několika klíčových slov, čísel v desítkové a hexadecimální soustavě a na zbylé texty, které bude považovat za identifikátory. 
Za oddělovače budou použity bílé znaky, jako jsou mezera, tabelátor, konec řádku a dále znaky rovnítka, čárky a středníku.
Mezera, tabelátor a znaky konce řádku budou oddělovat jednotlivé lexikální elementy. 
Čárka, rovnítko a středník budou plnit funkci oddělovačů a zároveň budou tokeny.
Výstupem programu bude výpis charakteristik všech tokenů.
\\

%\newpage
Příklad rozboru\\

Vstupní text:

\begin{verbatim}
  begin 
  
  a,b=100;
  
  c=0x10;
  
  end;
\end{verbatim}

Výpis tokenů
\begin{verbatim}
KeyToken{klicoveSlovo=KeyWord{key=begin}}
SeparatorToken{Separator{bílý znak}}
IdentifierToken{a}
SeparatorToken{Separator{čárka}}
IdentifierToken{b}
SeparatorToken{Separator{rovná se}}
NumberToken{value=100}
SeparatorToken{Separator{středník}}
SeparatorToken{Separator{bílý znak}}
IdentifierToken{c}
SeparatorToken{Separator{rovná se}}
NumberToken{value=16}
SeparatorToken{Separator{středník}}
SeparatorToken{Separator{bílý znak}}
KeyToken{klicoveSlovo=KeyWord{key=end}}
SeparatorToken{Separator{středník}}
SeparatorToken{Separator{bílý znak}}
\end{verbatim}


